\begin{verbatim}
#include "SIM900.h"
#include "sms.h"
#include "SoftwareSerial.h"
#include <TinyGPS++.h>
#include <PString.h>
int led=3;
TinyGPSPlus gps;
char buffer[160];
char smsbuffer[160];
char n[20];
unsigned long last = 0UL;
SMSGSM sms;
String kirim="+6281946052072";
PString str(buffer, sizeof(buffer));
void setup(){
  pinMode(led,OUTPUT);
  Serial.begin(9600);
  if (gsm.begin(9600)){
    
    sms.SendSMS("+6285975371023", "https://www.google.com/maps/@-6.9165056,107.5726355,13z");
    delsms();
   digitalWrite(led,HIGH);
  delay(1000); 
  digitalWrite(led,LOW);
  delay(1000);
  digitalWrite(led,HIGH);
  delay(1000); 
  digitalWrite(led,LOW);
  delay(1000);
  digitalWrite(led,HIGH);
  delay(1000); 
  digitalWrite(led,LOW);
  delay(1000);
  }
  
  
}

void loop(){
  
 
    kirim="+6285975371023";
   
  while (Serial.available() > 0)
   if( gps.encode(Serial.read()))
   info();
  
}
void(* resetFunc) (void) = 0;

void delsms()
{
  for (int i=0; i<10; i++)
  {  
      int pos=sms.IsSMSPresent(SMS_ALL);
      if (pos!=0)
      {
        if (sms.DeleteSMS(pos)==1){}else{}
      }
  }
}

void info(){
  str.begin();
    str.print("https://maps.google.com/maps?q=-6.876175,107.576080");
    str.print(gps.location.lat(), 6);
    str.print(F(","));
    str.print(gps.location.lng(), 6);
    kirim=str;
   
 
    
  int pos=0;
  pos=sms.IsSMSPresent(SMS_ALL);
  if(pos){
    sms.GetSMS(pos,n,smsbuffer,100);
    if(!strcmp(smsbuffer,"track")){
         digitalWrite(led,HIGH);
 
 
      str.begin();
      str.print(kirim);
      sms.SendSMS(n,buffer);
       digitalWrite(led,LOW); 
    }
    if(!strcmp(smsbuffer,"reset")){
      
      sms.SendSMS(n,"GPS TRACKER RESET");
      delay(5000);
      resetFunc(); 
    }
    delsms();
  }
  
  
} 
\end{verbatim}

\begin{verbatim}

#include <SoftwareSerial.h>

#include <TinyGPS++.h>

/*
 This example uses software serial and the TinyGPS++ library by Mikal Hart
 Based on TinyGPSPlus/DeviceExample.ino by Mikal Hart
 Modified by acavis
*/

// Choose two Arduino pins to use for software serial
// The GPS Shield uses D2 and D3 by default when in DLINE mode
int RXPin = 2;
int TXPin = 3;

// The Skytaq EM-506 GPS module included in the GPS Shield Kit
// uses 4800 baud by default
int GPSBaud = 4800;

// Create a TinyGPS++ object called "gps"
TinyGPSPlus gps;

// Create a software serial port called "gpsSerial"
SoftwareSerial gpsSerial(RXPin, TXPin);

void setup()
{
  // Start the Arduino hardware serial port at 9600 baud
  Serial.begin(9600);

  // Start the software serial port at the GPS's default baud
  gpsSerial.begin(GPSBaud);

  Serial.println(F("DeviceExample.ino"));
  Serial.println(F("A simple demonstration of TinyGPS++ with an attached GPS module"));
  Serial.print(F("Testing TinyGPS++ library v. ")); Serial.println(TinyGPSPlus::libraryVersion());
  Serial.println(F("by Mikal Hart"));
  Serial.println();
}

void loop()
{
  // This sketch displays information every time a new sentence is correctly encoded.
  while (gpsSerial.available() > 0)
    if (gps.encode(gpsSerial.read()))
      displayInfo();

  // If 5000 milliseconds pass and there are no characters coming in
  // over the software serial port, show a "No GPS detected" error
  if (millis() > 5000 && gps.charsProcessed() < 10)
  {
    Serial.println(F("No GPS detected"));
    while(true);
  }
}
 
void displayInfo()
{
  Serial.print(F("Location: ")); 
  if (gps.location.isValid())
  {
    Serial.print(gps.location.lat(), 6);
    Serial.print(F(","));
    Serial.print(gps.location.lng(), 6);
  }
  else
  {
    Serial.print(F("INVALID"));
  }

  Serial.print(F("  Date/Time: "));
  if (gps.date.isValid())
  {
    Serial.print(gps.date.month());
    Serial.print(F("/"));
    Serial.print(gps.date.day());
    Serial.print(F("/"));
    Serial.print(gps.date.year());
  }
  else
  {
    Serial.print(F("INVALID"));
  }

  Serial.print(F(" "));
  if (gps.time.isValid())
  {
    if (gps.time.hour() < 10) Serial.print(F("0"));
    Serial.print(gps.time.hour());
    Serial.print(F(":"));
    if (gps.time.minute() < 10) Serial.print(F("0"));
    Serial.print(gps.time.minute());
    Serial.print(F(":"));
    if (gps.time.second() < 10) Serial.print(F("0"));
    Serial.print(gps.time.second());
    Serial.print(F("."));
    if (gps.time.centisecond() < 10) Serial.print(F("0"));
    Serial.print(gps.time.centisecond());
  }
  else
  {
    Serial.print(F("INVALID"));
  }

  Serial.println();
}
\end{verbatim}